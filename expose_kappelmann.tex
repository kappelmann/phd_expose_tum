% \documentclass{article}
\documentclass[a4paper, 12pt]{article}

\usepackage{proof}
\usepackage[english]{babel}
\usepackage[utf8]{inputenc}
\usepackage{polski}
\usepackage[final]{microtype}
\usepackage[bottom]{footmisc}

% paper margins
\usepackage[margin=0.75in]{geometry}
\pagestyle{plain}
\usepackage{xspace}

% multiple colums
\usepackage{multicol}

%Title page settings
\usepackage[affil-it]{authblk}

% Title of document
\title{\textbf{PhD Exposé -- Soft Type Systems For Interactive Theorem Proving}}
% Author
\author{Kevin Kappelmann}
% \affil{Chair for Logic and Verification}
\date{\vspace{-5ex}}

%% Bibliography
\usepackage[numbers]{natbib}
% \usepackage[%
% backend=bibtex,
% url=true,
%style=alphabetic,
% maxnames=4,
% minnames=3,
% maxbibnames=99,
% giveninits,
% uniquename=init]{biblatex}

%% Graphics
\usepackage{xcolor}
\usepackage{graphicx}
\usepackage{tikz}
\usepackage{float}
\usepackage{standalone}
\usetikzlibrary{quotes,angles,calc,arrows.meta,positioning,automata}
\tikzset{initial text={}}
\usepackage{tikz-3dplot}
\usepackage{subcaption}

%% Hyperlinks
\usepackage{url}
\definecolor{pastelblue}{RGB}{0,72,205}
\usepackage[hidelinks]{hyperref}
\hypersetup{
 colorlinks,
 linktoc=page,
 linkcolor=pastelblue,
 citecolor=pastelblue,
 urlcolor=pastelblue
}

\usepackage{enumitem}
% \newlist{listnointend}{enumerate}{1}
% \setlist[listnointend,1]{
  % label={\arabic*},
  % leftmargin=*,
  % align=left,
  % itemindent=10pt
% }

%% Coding
\usepackage[outputdir=build]{minted}

% Math packages
\usepackage{amsmath}
\usepackage{mathtools}
\usepackage{amssymb}
\usepackage{mathpartir}
\usepackage{wasysym}
\usepackage{stmaryrd}
\usepackage{centernot}

% Proof system
\usepackage{amsthm}
%% References
\usepackage[capitalize,sort,compress,noabbrev]{cleveref}
\crefdefaultlabelformat{#2\textup{#1}#3}
\Crefname{thm}{Theorem}{Theorems}
\Crefname{lem}{Lemma}{Lemmas}
\Crefname{prop}{Proposition}{Propositions}
\Crefname{cor}{Corollary}{Corollary}
\Crefname{defn}{Definition}{Definitions}
\Crefname{exmpl}{Example}{Examples}
\Crefname{rmk}{Remark}{Remarks}
\Crefname{propy}{Property}{Properties}
\Crefname{claim}{Claim}{Claim}
\Crefname{assmlisti}{Assumption}{Assumptions}
\Crefname{invarlisti}{Invariant}{Invariants}

% spacing fixes for proof system
\makeatletter
% \def\thm@space@setup{%
  % \thm@preskip=\parskip \thm@postskip=0pt
% }
\makeatother
\theoremstyle{plain}
\newtheorem{thm}[equation]{Theorem}
\newtheorem{lem}[equation]{Lemma}
\newtheorem{prop}[equation]{Proposition}
\newtheorem{cor}[equation]{Corollary}
\theoremstyle{definition}
\newtheorem{defn}[equation]{Definition}
\newtheorem{exmpl}[equation]{Example}
\newtheorem*{rmk}{Remark}
\makeatletter

% Custom commands

% Quotes with author
\usepackage{xparse}

\let\oldquote\quote
\let\endoldquote\endquote

\RenewDocumentEnvironment{quote}{o}
  {\oldquote}
  {\IfValueT{#1}{\par\nobreak
   \hfill--- #1}\endoldquote\addvspace{\smallskipamount}}

%------------------------------------------------------------------------------
\begin{document}
% \pagenumbering{gobble}
\maketitle

\abstract{
Most interactive theorem provers
are based on a rigid kind of type theory.
In such systems, typing rules
are fixed once and for all
and users are bound to formlise their work
subject to these rules.
An alternative is to use soft type systems.
In such systems, user can specify new typing rules
at any time without threatening soundness of the prover.
Soft types have the potential to make
proof assistants more flexible
and expressive.
Yet, they are heavily underexplored.
In my PhD project,
I want to assess the practicality of soft type systems
and lay the foundations for a modern soft type
framework for interactive theorem provers.
}
% \tableofcontents

% \newpage
\pagenumbering{arabic}

\section{Introduction and Motivation}\label{sec:intro}

The establishment of mathematical truths rests on
the discovery of proofs.
In simple terms, one may think of a proof
as a coherent chain of logical arguments
leading from a collection of premises to a conclusion.
This, however, leads to many questions:
When is an argument ``logical''?
What rules and premises are there to begin with?
Some of the most distinguished mathematicians
intensively debated these questions in the beginning of the 20th century,
and the rigorous, axiomatic foundations of modern mathematics are the result of these debates~\cite{hilbertprogram}.

However, on a daily basis,
contemporary scientists do not publish
complete formal proofs with respect to some axiomatic foundation,
which often requires tedious low-level reasoning,
but rather use reasoning principles of a higher level.
A proof written this way is accepted if the community agrees that the reasoning at hand can, in theory,
be translated to an accepted axiomatic basis.
In practice, however,
this translation is not spelled out,
and the proof remains susceptible to human error.
% despite all scientific rigour that went into it.

Interactive theorem provers (also known as proof assistants),
such as Isabelle~\cite{isabelle}, Coq~\cite{coq}, and Lean~\cite{lean},
try to close this gap by providing
a computer-checked form of mathematical reasoning.
At their core,
they all rely on the axiomatic method of mathematics.
As such,
they fix a set of axioms on top of which users
can axiomatise and define new concepts and prove theorems.

Most interactive theorem provers
are based on some form of \emph{type theory}.
Simply speaking,
types act as categorisations of mathematical values and structures.
The natural numbers ($\mathbb{N}$),
functions on integers ($\mathbb{Z}\Rightarrow \mathbb{Z}$),
and the collection of $m\times n$ matrices in a given domain ($D^{m\times n}$) are examples of types.

Systems based on type theory
use a set of \emph{typing rules} and a \emph{type checker}.
The typing rules are the axioms of the type system,
specifying which expressions may belong to each type.
They are usually stated in a declarative form.
The type checker is the algorithmic counterpart of these declarative rules.
It takes a term $t$ and a type $T$ and
checks, according to the typing rules,
whether $t$ is of type $T$, which we denote by $t : T$.

Type systems are a very helpful mechanism,
preventing users from common mistakes,
such as applying a function to a value outside its domain.
However, as mentioned by \citet{krauss},
they deviate from the de facto standard
in mathematics,
where a set-theoretic foundation is taken for granted.
Moreover, virtually all major proof assistants
take their set of typing rules and their type checker
as a fixed primitive.
When a mathematical concept is not easily expressible using these primitives,
types can become a serious limitation
(see \citet{krauss} for some examples).

While many of these limitations could be addressed
by appropriately modifying the type system at hand,
this rarely happens in practice:
First, the type system builds the axiomatic basis of said proof assistants.
Changing it hence raises meta-theoretical soundness concerns.
Second, modifying the core of such systems
requires significant implementation work.
Third, no set of typing rules is perfect
and users thus may wish to switch from one set of rules to another one
without introducing incompatible foundations.

I propose to address these limitations
by developing a \emph{soft type}\footnote{The term originates from the programming language community~\cite{cartwright},
where it has a different meaning: it is the principle of statically type-checking as much as possible while inserting run-time checks for what cannot be statically checked.}
framework for proof assistants.
Unlike traditional ``hard'' type systems,
soft type systems do not take the notion of a type as primitive.
Instead, they reduce it to existing concepts of the underlying logic.
In their most simple form,
soft types can be defined as predicates on terms:
a term $t$ is of soft type $T$ if and only if $T(t)$.
Moreover, a typing rule simply becomes a proven inference rule in the underlying logic.
This way, types and typing rules become first-class objects,
and type checking nothing more than
theorem proving inside the given logic.

As explained by \citet{krauss},
soft type systems provide an extremely flexible and powerful framework:
user can add and remove types and typing rules anytime,
and they can do so without changing the foundation of the system.
Treating types as predicates also more closely
resembles the way mathematicians
think about types in set-theoretic frameworks,
as explicated by \citet{harrison}.
For example,
given the set of natural numbers $\mathbb{N}$,
one can define the type of natural numbers as the predicate
$\mathtt{Nat}\ n \coloneqq n \in \mathbb{N}$.
Moreover,
soft types allow for an inclusive
interpretation of subtyping (in which $t : A$ and $A <: B$ implies $t : B$)
rather than
an coercive one (where terms of a subtype need to be lifted with coercion functions).
As an example, $n : \mathbb{N}$ does not imply $n : \mathbb{Z}$ in usual hard type systems
but the implication does hold in a suitable soft type system based on set theory.
% This is particularly interesting in the case of set-theoretic number systems:
% there we have $\mathbb{N}\subseteq\mathbb{Z}$,
% but $n : \mathbb{N}$ does not imply $n : \mathbb{Z}$ in corresponding hard type systems.

\section{State of the Art}

Interactive theorem provers can broadly be
classified into
\begin{enumerate}
\item the higher-order logic provers
based on simply-typed lambda calculus
(e.g.\ HOL-Light, Isabelle/HOL),
\item the dependent type provers (e.g.\ Agda, Coq, Lean), and
\item the provers based on traditional set theory (e.g.\ Isabelle/ZF, Metamath, Mizar).
\end{enumerate}
By design, proof assistants of the first two categories are based on hard type systems of differing strengths and weaknesses.
Though they proved quite successful and
received more attention than their set-theoretic counterparts,
they all suffer from the limitations sketched in \cref{sec:intro}.
Proof assistants of the third category,
in contrast,
lend themselves particularly well to soft type systems.
However, most such provers stay in a completely untyped or at best very restricted soft type framework.

A notable exception is Mizar~\cite{mizar}.
Mizar is based on an untyped set theory but nevertheless supports the notion of types.
Mizar users can define new types
by defining predicates on sets.
Its typing rules are not fixed axiomatically
but rather are specified and proven to be sound with
the help of the user in a controlled manner.
Mizar pioneered the realm of soft types and has a remarkable
library of formalised mathematics~\cite{mml}.
However, Mizar falls short in important aspects:
\begin{enumerate}
  \item It uses a fixed type checker
    which is not proof-producing but part of the trusted kernel~\cite{wiedijk}.
    As such, it cannot be modified without raising meta-theoretical soundness concerns.
    Moreover, Mizar is closed-source, and as such,
    users have no possibility to modify or replace the type checker.
  \item It only accepts typing rules adhering to a strict format (so-called ``registrations'').
  \item It is neither interactive nor programmable;
    as such, it is more akin to being ``just'' a proof verifier rather than an interactive theorem prover.
  \item It uses idiosyncratic type-theoretic concepts and terminology,
    rendering its usage difficult to both
    computer scientists and mathematicians.
\end{enumerate}

To the best of my knowledge,
there is no other significant work towards
a rich soft type framework for proof assistants.

\section{Objectives and Research Questions}

My objective is to lay the foundations for a modern soft type framework for interactive theorem provers.
First, this requires a deep understanding of state-of-the-art hard type systems
and formulating appropriate generalisations thereof that can deal with the more flexible soft type approach.
Second, this requires proving practicality of the approach and contrasting it with
existing frameworks employed by proof assistants
by means of a workable implementation and appropriate case studies.

As part of this endeavour, a number of different research questions arise, such as:
\begin{itemize}
\item Which requirements should a modern soft type system meet?
  Which concepts known from hard type systems should be integrated?
  Which ones have to be generalised?
  Which additional concepts have to be introduced?
\item Can we build a soft type system which is flexible, predictable, and sufficiently efficient?
\item Which structures (e.g.\ sets, lambda terms) lend themselves well for soft type systems? Which soft typing rules form a practical basis for these structures?
\item Some concepts, such as function spaces, may be expressible in both the underlying logical foundation (e.g.\ simply-typed lambda calculus) and the structures on top of which the soft type system is employed (e.g.\ sets).
  Can we efficiently move between both worlds?
\item How can we integrate soft types in existing proof assistants in a user-friendly manner?
\item Can soft types effectively help us expressing concepts that usually are impossible, or at least difficult,
  to formulate using just the underlying logic of a given proof assistant
  (e.g.\ using dependent soft types in HOL provers)?
\end{itemize}
Given the complexity and extent of modern research in type theory and proof assistants,
I acknowledge that it is not realistic to answer all of these questions
in my PhD project,
but I aim to take major steps in this direction.

\section{Methodology and Project Plan}

Krauss and Chen~\cite{isabelleset} started a first prototype of
a soft type system in Isabelle/HOL,
which I joined at the beginning of my PhD.
We built a simple soft type framework in which soft types
are identified by predicates on terms.
To test the framework,
we first axiomatised a Tarski–Grothendieck set theory
inside Isabelle/HOL.
Relative consistency of such a theory
has been shown by \citet{brown2019higher}.
Second, we built a basic untyped library for this
set theory, inspired by the work in Isabelle/ZF~\cite{isabellezf}.
Third, we combined the untyped set theory with
our soft type framework and built a small mathematical library that uses the idea of soft types on sets.
This library is called Isabelle/Set and is continuously updated on GitHub~\cite{isabelleset}.

Let me explain our choices.
\begin{itemize}
  \item We use Isabelle because:
    \begin{enumerate}
      \item It supports multiple object logics.
        This allows us to test the flexibility and practicality of the soft type framework with respect to different underlying foundations (e.g.\ higher-order and first-order logic).
      \item It is one of the most prominent proof assistants, offering large proof libraries and good automation and programming documentation.
      \item It is co-developed at the Technical University of Munich and both my supervisor Prof.\ Nipkow and I have experience using it.
    \end{enumerate}
  \item We built a library in Isabelle/HOL rather than Isabelle/ZF because:
    \begin{enumerate}
      \item The first-order nature of Isabelle/ZF prevents us from expressing
        soft types that require higher-order quantification (such as the second-order function \makebox{$(\mathbb{N}\Rightarrow\alpha)\Rightarrow\alpha$}).
    While in theory, such functions can be internalised in the set theory, this is not done in Isabelle/ZF and rewriting it to do so requires substantial work.
  \item Isabelle/HOL provides mechanisms which can prove valuable to the development of a soft type system.
    For example, a generalisation of the transfer and lifting packages~\cite{lifting} may be used to move between related softly-typed structures within the set theory.
    \end{enumerate}
  \item The choice of a Tarski-Grothendieck set theory is not crucial and could be replaced by other sufficiently strong theories.
\end{itemize}

With this prototype at hand,
we were able test the feasibility of the approach and discovered new challenges.
At this point, Krauss and soon after also Chen stepped
back from the project and I became the sole researcher and developer of the project.

Following further analysis and an extensive phase of literature research,
I identified the following problems:
\begin{itemize}
  \item The current type checker is not strong enough:
    It is based on a simplification of the algorithm used in Mizar.
    However, the typing rules we use in Isabelle/Set are not restricted
    in the same way as Mizar's typing rules are.
    As such, many type checking problems that user's expect to
    be trivial cannot be discharged and are left for manual proof.
  \item The current type checker and synthesiser are two separate algorithms.
    This approach differs from modern programming languages and proof assistants,
    which use \emph{bidirectional typing} instead.
    Bidirectional typing supports features for which pure type synthesis
    is undecidable~\cite{bidirectional}.
    This reduces the type annotation burden put on users.
  \item The current type checker and synthesiser are not flexible enough:
    the latter expects a certain form of function types and both
    can barely be extended or guided by users without touching the Isabelle/ML code.
  \item We developed a \emph{set extension} mechanism that,
    for example, allows us to construct a set-theoretic number system
    with the wanted inclusions $\mathbb{N}\subset\mathbb{Z}\subset\mathbb{Q}\subset\mathbb{R}\subset\mathbb{C}$.
    As part of this process, a set extension takes a set $A$ and
    creates a new set $A'$ together with a set-isomorphism $f : A\to A'$.
    One then wishes to lift existing definitions and theorems on $A$ to $A'$ using $f$.
    However, the current lifting package of Isabelle/HOL does not support this use case since $f$ is not right-total and right-unique on the underlying Isabelle/Pure type of $A'$.
  \item
    % The type class mechanism of Isabelle~\cite{isatypeclasses} is weaker than the type class mechanisms of Coq and Lean.
    % We can obtain the same expressivity in Isabelle
    % by encoding type classes using soft types and sets.
    In certain cases,
    users must\footnote{for example, when encoding type classes as sets} -- or may wish to -- internalise Isabelle/Pure functions to set-theoretic functions.
    At the moment, users have to do this by hand.
    Automation would greatly improve the user experience.
\end{itemize}
In the next steps of my PhD, I propose to address these problems as follows:
\begin{itemize}
  \item The type checker and synthesiser are unified into a bidirectional typing framework.
    Inspired by state-of-the-art work on hard type systems
    \cite{de2015elaboration,guidi2019implementing,mazzoli2016type,tassi2012bi,vytiniotis2011outsidein}
    and based on my experience with Isabelle/Set,
    I propose to build the framework on a constraint logic programming basis.
    The framework will be extensible -- users can register new constraint solvers at any time --
    and controllable -- users can add, remove, and adjust the priorities of typing rules.
    To my understanding,
    most hard type systems can be encoded as a constraint logic program.
    I plan to stress this by providing examples of such encodings using my framework.
  \item The lifting package of Isabelle/HOL can be generalised to support the cases arising as part of the set extension mechanism of Isabelle/Set.
    Together with Juli Gottfriedsen,
    I already generalised the required theory and proved its correctness in unpublished work.
    As a next step,
    a generalisation of the lifting algorithm will be implemented and its
    usability tested in a case study.
    As part of this study,
    I expect to categorise classes of subproblems that are generated when applying the generalised lifting algorithm.
    Finding appropriate automation to solve these subproblems follows as a next step.
    I moreover plan to investigate whether the lifting package can be useful when dealing with the internalisation problem mentioned above.
  \item Having both the new soft type framework and lifting package at hand,
    I plan to assess the practicality of the resulting system
    by conducting a formalisation case study that
    requires advanced type-theoretical concepts (dependent and intersection types, type classes, etc.).
    The formalisation will be contrasted with those of corresponding developments in other proof assistants.
    The results of this study will indicate future directions and possible improvements of the framework.
\end{itemize}
The complete project schedule of my PhD project can be found in TODO.

Finally, let me stress that,
while the practical implementation of my work will be done in the Isabelle proof assistant,
the theoretical knowledge gained during the project will be applicable to interactive theorem provers and the type theory community in general.

\newpage
\bibliographystyle{plainnat}
\bibliography{sources}

\end{document}

